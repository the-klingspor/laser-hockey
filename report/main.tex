\documentclass[11pt]{article}
\usepackage{booktabs}
\usepackage{times}
\usepackage[utf8]{inputenc} % allow utf-8 input
\usepackage[T1]{fontenc}    % use 8-bit T1 fonts
\usepackage{url}            % simple URL typesetting
\usepackage{graphicx}
\usepackage{color}
\usepackage{amsfonts}       % blackboard math symbols
\usepackage{amsmath}       % blackboard math symbols
\usepackage{amssymb}
\usepackage{float}

\usepackage{lipsum}

\usepackage{geometry}
\geometry{left=2.8cm,right=2.8cm,top=2.6cm,bottom=2.6cm}
\usepackage{fancyhdr}
\pagestyle{fancy}
\usepackage{hyperref}% should be the last package you include

\newcommand{\theteam}{}
\newcommand{\team}[1]{\def\theteam{#1}}


\fancyhead[L]{\theteam}
\fancyhead[R]{\thepage}
\cfoot{}

\setlength{\parindent}{0pt}

\team{Arizona Codeyotes: Joschka Strüber, Bálint Mucsányi, Enes Duran}
\title{RL-Course 2023: Final Project Report}
\author{\theteam}

\begin{document}
\maketitle

\section{Introduction}

In the final project of the Reinforcement Learning course of the University of Tübingen offered in Summer Semester 2023, we were tasked to implement agents with nontrivial contributions that successfully play against opponents with a wide variety of playstyles in the \texttt{laser-hockey} environment. Being a two-player game, the value of individual states can highly vary based on the opponent, highlighting the high aleatoric uncertainty (inherent noise in the data-generating process) that one has to face when developing robust agents. To solve this challenging task, we set up a team of three members, Joschka Strüber, Bálint Mucsányi, and Enes Duran, and decided to implement the following algorithms:
\begin{itemize}
    \item \textbf{Joschka Strüber}: the Twin Delayed Deep Deterministic policy gradient algorithm (TD3)~\cite{fujimoto2018:TD3};
    \item \textbf{Bálint Mucsányi}: the full Rainbow algorithm~\cite{Hessel2018:Rainbow};
    \item \textbf{Enes Duran}: the Soft Actor-Critic algorithm (SAC)~\cite{HaarnojaAbbeelLevine2018:SAC}.
\end{itemize}

We decided on these algorithms mainly because of their algorithmic differences while being notably performant. TD3 is a highly robust and widely used actor-critic method using a continuous action space. Rainbow is a similarly popular off-policy algorithm with a discrete action space that contains various extensions to the seminal Deep Q-Network (DQN) paper~\cite{mnih2015humanlevel}. SAC is an entropy-based off-policy algorithm demonstrating the effectiveness of balancing exploration and exploitation \cite{HaarnojaAbbeelLevine2018:SAC}. Understanding how these notably different algorithms behave in the \texttt{laser-hockey} environment and what components are most useful/harmful for the task proved to be an exciting and demanding problem.

\subsection{Environment}\label{subsec:environment}

The \texttt{laser-hockey} environment is a custom environment based on the \texttt{gymnasium}~\cite{towers_gymnasium_2023} Python package. Two agents play against each other, each controlling a simple hockey stick, trying to hit the hockey puck to score a goal. The environment is fully symmetric: the agents obtain perfectly mirrored states from the environment and therefore no side switching is needed when developing the algorithms. A particular state returned by the environment includes the $(x, y)$ coordinate of the players and the puck; the angle, angular velocity and linear velocity of the players; and the linear velocity of the puck. For the challenge, we were required to use the \texttt{keep-mode} variant of the environment in which the puck sticks to the hockey sticks and players can decide when to shoot the puck. In detail, the action space is a four-dimensional box $\in [-1, 1]^4$ containing the force applied in the $x$ and $y$ directions, the torque applied to the hockey stick, and the thresholded decision of shooting the puck or not. The default reward returned by the environment comprises the sparse reward $\in \{-10, 0, 10\}$ given for losing, having a draw, and winning, respectively; and the per-frame \texttt{reward\_closeness\_to\_puck} component that punishes being far away from the puck.

Our complete implementation, including code necessary for training on the tournament server, is available at TODO(Joschi, Bálint). To clearly separate the contributions of the team members, we provide a list of them below per person.

\medskip

\textbf{Bálint Mucsányi}: \texttt{Algorithm}, \texttt{Agent}, \texttt{ReplayBuffer}, \texttt{BaselineReward}, \texttt{MLP}, \texttt{Noisy Linear}, and \texttt{DistributionalReward} classes; complete Rainbow algorithm; action space augmentation; common utilities; training on the tournament server.

\medskip

\textbf{Joschka Strüber}: \texttt{SparseReward}, \texttt{SimpleReward}, and \texttt{PrioritizedReplayBuffer} classes; complete TD3 algorithm; code for evaluation; logging using \texttt{wandb}~\cite{wandb}; state space augmentation; pink noise.

\medskip

\textbf{Enes Duran}: Complete SAC algorithm and its variants.

\medskip

Next we discuss each algorithm in detail, including descriptions of the methods used and experimental evaluation.

%\section{Twin Delayed Deep Deterministic Policy Gradient}
\section{TD3}

\subsection{Methods}

Our first approach uses the off-policy algorithm Twin Delayed Deep Deterministic policy gradient \cite{fujimoto2018:TD3}. In the following section we will briefly introduce this approach along with additional techniques proposed to solve the \texttt{laser-hockey} environment. All presented ideas will be thoroughly evaluated and compared in Section~\ref{subsec:td3_eval}.

During the tournament we have used a bandit-based Upper Confidence Bound approach to select the agent that performs best against the unknown, opposing agents \cite{agrawal1995SampleMB}. Due to space constraints this section has been moved to the appendix in Section~\ref{subsec:ucb}.

\subsubsection{Twin Delayed Deep Deterministic Policy Gradient}

Twin Delayed Deep Deterministic policy gradient (TD3) is an actor-critic based reinforcement learning algorithm that improves upon the earlier DDPG approach \cite{fujimoto2018:TD3, lillicrap16ddpg}. TD3 has one neural network as actor $\pi_{\phi}$ and two critics $Q_{\theta_1}, Q_{\theta_2}$, that estimate the Q-function for state-action pairs. The algorithm uses continuous actions, making it particularly useful for our environment where precise actions such as turning and moving are vital for defending your goal and scoring. Being an off-policy algorithm, TD3 is very suitable for a two-player game where fine-tuning on the tournament data can be the deciding factor. 

Given an observation $s$ during training along, TD3 uses actor $\pi_{\phi}$ with exploration noise to choose the next action $a = \pi_{\phi} + \epsilon, \epsilon \sim \mathcal{N}(0, \sigma)$. After training, the actor can be used in a deterministic way without exploration noise. In each training step, a batch of $N$ transitions $(s, a, r, s')$ is sampled from replay buffer $\mathcal{D}$. To compute the current Q-values target policy smoothing regularization is used and the target network of the actor predicts the action for the next step. This is then used to compute the smoothed TD targets, whose mean squared error is minimized to optimize the critics $Q_{\theta_i}$:
\begin{align*}
    \Tilde{a} &= \pi_{\phi'}(s') + \epsilon, \quad \epsilon \sim \operatorname{clip}(\mathcal{N}(0, \Tilde{\sigma}), -c, c) \\
    \operatorname{TD}(s, a, \Tilde{a}, r, s') &= (r + \gamma \min_{i=1,2} Q_{\theta'_i}(s', \Tilde{a})) - Q_{\theta_i}(s, a),
\end{align*}

where the target policy smoothing with clipped noise ensures that similar actions have similar Q-values. Taking the minimum of the two critics severely reduces overestimation of the Q-function. Next, the actor is updated using the deterministic policy gradient in a delayed way every $d$'th critic update:
\begin{equation*}
    \triangledown_{\phi} J(\phi) = \frac{1}{N} \sum \triangledown_a Q_{\theta_1}(s,a)|_{a=\pi_{\phi}}(s) \triangledown_{\phi} \pi_{\phi}(s)
\end{equation*}

 The delayed policy update reduces policy degradation caused by bad value estimates. All three target networks $\pi_{\phi'}$, $Q_{\theta'_1}$ and $Q_{\theta'_2}$ are updated with Polyak averaging using parameter $\tau \in (0,1)$.

\subsubsection{Reward Optimization and State Space Augmentation}\label{subsubsec:state_space}

As highlighted in Section~\ref{subsec:environment}, the default reward of the \texttt{laser-hockey} environment returns a canonical, sparse reward $\in \{-10, 0, 10\}$ depending on the outcome of a game with a modification using the \texttt{closeness\_to\_puck}. However, learning complex behavior with reinforcement learning agents using only sparse rewards is inherently difficult. In the case of \texttt{laser-hockey} the high frame rate causes large amounts of steps between the action that lead to shooting a goal and the moment the reward is received. 

%This makes learning the right Q value of state action pairs difficult.

To deal with this problem, the environment provides four points of information that can be used for proxy rewards: \texttt{winner} $\in \{-1, 0, 1\}$, \texttt{closeness\_to\_puck}, which punishes being too far away from the puck, $\texttt{touch\_puck}$, which rewards taking possession of the ball, and finally \texttt{puck\_direction}, that rewards if the puck is moving in the right direction.

We evaluated 1000 games of the strong \texttt{BasicOpponent} playing against itself and used the mean cumulative absolute info values as weights for our \texttt{weighted\_reward} = \texttt{sparse\_reward} $+\, 0.05 \cdot \texttt{closeness\_to\_puck}$ $+\, 1.0 \cdot \texttt{touch\_puck}$ $+\, 3.0 \cdot \texttt{puck\_direction}$. This weighting provides feedback to learn useful behaviour quickly, such as moving towards and touching the puck. Yet, the weights are small enough to ensure that winning the game is crucial to maximizing the reward.

The environment provides eighteen dimensional observations containing the coordinates, speed and orientation of both players and the puck. For more high-level information about the state of the game, we augment the observation space with the euclidean distance between all reasonable combinations of the two agents, the puck and both goals. This increases the dimension of the observation space to 27. 

\subsubsection{Pink Noise}

As TD3 is an off-policy algorithm, it is possible to replace the Gaussian distribution that is commonly used to sample action noise with temporally correlated noise, such as Ornstein-Uhlenbeck noise \cite{uhlenbeck30noise}, leading to a better state space coverage. Recently, it has been shown that temporally correlated Pink Noise is a good default choice that often outperforms other distributions and rarely performs statistically significantly worse in classic reinforcement learning settings \cite{eberhard-2023-pink}. Pink Noise is colored noise where the power spectral density is proportional to $f^{-1}$.

\subsection{Experimental Evaluation}\label{subsec:td3_eval}

To find the best approach for the final tournament, we conducted extensive experiments on both seen and unseen agents. The first experiments included different reward and architectures against the weak and strong \texttt{BasicOpponent}. After finding a setup that is good enough to beat the opponents it was trained against, we wanted to ensure that our agent is able to beat stronger, unseen agents. For this we did further experiments with more training steps, self-training and advanced techniques such as Pink Noise. 

\begin{table}[]
    \centering
    \caption{Ablation study of TD3 showing the average win percentages and standard deviations across five seeds against the weak and strong \texttt{BasicOpponents}.}
    \label{tab:td3_base_experiment}
    \begin{tabular}{|c|c|c|c||l|l|} 
    \hline
    \textbf{Reward} & \textbf{Architecture} & \textbf{Activation} & \textbf{State Augm.} & \textbf{Weak Opponent} & \textbf{Strong Opponent} \\    \hline
    Sparse & 256 $\times$ 256 & ReLU & / & $38.90\% \pm 22.19$ & $33.14\% \pm 15.75$ \\ 
    Weighted & 256 $\times$ 256 & ReLU & / & $89.02\% \pm 20.37$ & $89.19\% \pm 19.05$ \\
    Weighted & 256 $\times$ 256 & ReLU & \texttt{distance} & $\mathbf{99.06\% \pm 0.79}$ & $\mathbf{98.22\% \pm 0.97}$ \\
    \hline
    \end{tabular}
\end{table}

\subsubsection{Base Experiments}

Table~\ref{tab:td3_base_experiment} shows the ablation study to find the best setup for TD3 with the reward and state space augmentation. The full ablation study including the architecture and activation functions can be found in Table~\ref{tab:td3_base_experiment_full} in the Appendix. All experiments were conducted on the same set of five seeds and hyperparameters, to balance out the natural randomness of the environment and training process as suggested by \cite{henderson18matters}. 

Every training run consisted of 3 million steps, where the opponent of each game was uniformly sampled from the weak and strong \texttt{BasicOpponents} after an initial starting period of 50,000 steps against the weak only. The final evaluation contains the win percentage against each opponent averaged across 1,000 games, along with one standard deviation. The hyperparameters can be found in Table~\ref{tab:td3_params} and were optimized for the experiment with the weighted reward, $(256 \times 256)$, ReLU activation function and without state space augmentation. We are aware that other settings might have performed better with a more thorough, specific optimization that was infeasible, because of compute constraints. 

In the first experiment of the ablation we compare the sparse reward against the weighted reward. We can clearly see that the modified reward results in much better results for the amount of time steps trained. Next, we added the \texttt{distance} state augmentation, resulting in an approach that is consistently able to converge in a small number of steps and beat both \texttt{BasicOpponents}. Since we were very close to playing perfectly against these agents, we decided to switch to a more difficult setup to perform further optimization. 

\subsubsection{Advanced Experiments}

When playing against other trained agents that were able to always beat the \texttt{BasicOpponents}, it became apparent that just training against those opponents results in severe overfitting. While they are perfectly solving the environment they were trained in, they don't generalize to unseen opposing agents and their play styles are easily exploited. For this reason, we picked four agents from the experiments with the $(128 \times 128)$ and $(400 \times 300)$ architectures that beat the two \texttt{BasicOpponents} consistently. All experiments were trained for twelve million steps using the same hyperparameters as before. The experiments can be seen in Figure~\ref{fig:td3_experiments}. 

%As all agents from all approaches across every seed were able to have an average win percentage of more than $95\%$ they were not included in the evaluation. We evaluate our agents every $100,000$ steps against for $1000$ games against the same set of opponents. Due to computational constraints, these experiments were done using only three seeds. 

The left experiment includes our best setup from the base experiments, another version where we replace the Gaussian action noise with Pink Noise and both of these in combination with self-training. We start by playing just against both \texttt{BasicOpponents} until a win percentage of $90\%$ against the stronger one is reached. At that point, we start adding frozen copies of the agent to a pool of fifteen self-training opponents. Every $100,000$ steps the oldest version of them is removed and replaced. The best results would have been reached after eight million training steps and further training resulted in overfitting. Using a comparable set of unseen agents could have been used for unbiased early stopping.

The second experiment on the right shows the results of self-training along with training against other strong, previously trained agents as well as a version that uses four stacked observations from consecutive time steps. If we use training against pretrained opponents, we additionally add a set of fifteen other strong, unseen agents to the set of opponents to uniformly sample from. They are added once one million steps with self-training have been done. These agents are unrelated to the ones we evaluate against.

We can clearly see that self-training and particularly training against other strong agents helps with generalization. In all experiments Gaussian noise performed better than Pink Noise. We suspect that global state space exploration, as performed by Pink Noise, may be harmful in the \texttt{laser-hockey} environment, because the default spawn position in the middle of the field provides a good chance of defending the goal. Training against other trained agents along with self-training on the other hand resulted in the best results, leading to quick convergence, while also preventing overfitting.

\begin{figure}
    \centering
    \includegraphics[width=\linewidth]{gfx/advanced_experiments_td3.pdf}
    \caption{Fraction of games won for different TD3 setups against a set of unseen agents. Lines correspond to the mean performance; shaded areas correspond to one standard deviation across three seeds.}
    \label{fig:td3_experiments}
\end{figure}

\section{Rainbow}

The Rainbow algorithm~\cite{Hessel2018:Rainbow} is a collection of improvements upon the seminal Deep Q-Network (DQN) algorithm~\cite{mnih2015humanlevel} that revolutionized the use of deep learning architectures for reinforcement learning problems. \ref{subsec:rainbow_methods} introduces all techniques used in Rainbow, and \ref{subsec:rainbow_experimental} gives a detailed evaluation of the effect of individual components and further modifications made to the learning setup.

\subsection{Methods}\label{subsec:rainbow_methods}

Deep reinforcement learning is, as of 2023, the dominant approach to reinforcement learning problems. This paradigm introduces the use of deep neural networks for function approximation, giving powerful representations for policies $\pi_\theta(s, a)$ or values $\{\hat{v}(s, w)$, $\hat{q}(s, a, w)\}$ or both when the state space or the action space is intractably large or continuous.

In DQN, only the Q-value function is explicitly represented as $\hat{q}(s, a; w)$ and we aim to solve the Bellman optimality equation by minimizing the loss $\mathcal{L}(w) = \mathbb{E}_{s, a, r, s', d \sim \mathcal{D}}\left[\text{TD}(s, a, r, s', d)^2\right]$ where $\mathcal{D}$ represents a uniform replay buffer, and the tuple $(s, a, r, s', d)$ is collected and stored in $\mathcal{D}$ by an arbitrary off-policy agent (e.g., $\epsilon$-greedy w.r.t. $\hat{q}(\cdot, \cdot; w)$). The TD error is defined as $\text{TD}(s, a, r, s', d) = r + d\gamma \max_{a'} \hat{q}\left(s', a'; w^\text{target}\right) - \hat{q}(s, a; w)$ with $d = [\![ s' \text{ is terminal} ]\!]$. Using the notation introduced in the lecture, $w^\text{target}$ refers to the target network weights that are updated either by setting $w^\text{target} \gets w$ every $k$ gradient steps or by using an exponential moving average (the Polyak update).

Let us introduce the modifications made to the Markov Decision Process (MDP) to facilitate learning and make the agents more performant.

\subsubsection{Modifications to the Markov Decision Process}\label{subsubsec:mdp}

First, we consider a new reward discussed in~\ref{subsubsec:state_space} that leverages all parts of the information dictionary and leads to much faster convergence (\ref{subsubsec:base}). We also augment the observation space with pairwise distances between the opponents, the puck, and the goals, making the state space 27-dimensional. We further experiment with observation stacking where one state in the MDP contains $n$ consecutive observations/frames. Finally, we compare a basic 8-dimensional action space with an advanced 24-dimensional one, both listed in Appendix~\ref{appendix:rainbow}. Let us discuss the Rainbow algorithm's extensions compared to the base DQN algorithm.

\subsubsection{Prioritized Experience Replay}

We first consider the Prioritized Experience Replay (PER) extension~\cite{schaul2016prioritized} as it showed a consistent advantage over the regular experience replay using uniform sampling during our preliminary experiments. To sample transitions the agent can learn a lot from, PER samples them proportionally to their loss value: $P((s, a, r, s', n)) \propto \left(\begin{cases}0.5 \cdot \text{TD}(s, a, r, s', d)^2 & \text{if } |\text{TD}(s, a, r, s', d)| < 1 \\ |\text{TD}(s, a, r, s', d)| - 0.5 & \text{otherwise}\end{cases}\right)^{\alpha}$ where $\alpha$ is a hyperparameter we set to $0.5$.

\subsubsection{Dueling Networks}

We consider the Dueling Networks extension~\cite{wang2016:DDQN} next. Our initial experiments showed it to be the most important addition for the \texttt{laser-hockey} environment (\ref{subsec:rainbow_experimental}). In Dueling Networks, we separate the Q-value computation into an advantage and a value stream. These share a common feature extractor $f(s; \xi)$ that is used to calculate the advantage and value streams as $\hat{a}(f(s; \xi), a; \psi)$ and $\hat{v}(f(s; \xi); \eta)$. Finally, the Q-value is given by $\hat{q}(s, a; w) = \hat{v}(f(s; \xi); \eta) + \hat{a}(f(s; \xi), a; \psi) - \frac{1}{|\mathcal{A}|}\sum_{a'} \hat{a}(f(s; \xi), a'; \psi)$ where $\mathcal{A}$ is the discrete set of actions and $w = \{\xi, \eta, \psi\}$.

\subsubsection{Double Q-Learning with Multi-Step Returns}\label{subsubsec:double}

As seen in the lecture, there is an inherent overestimation bias in Q-Learning due to the max operator in the TD target. In particular, as the Q-values are noisy during training, we also take the maximum over the added noise, which results in a systematic bias. Double Q-Learning~\cite{vanhasselt2015deep} mitigates this bias by separating the selection of the maximizing action using the online network $\hat{q}(\cdot, \cdot; w)$ and the prediction of the value corresponding to the maximizing action using the target network $\hat{q}(\cdot, \cdot; w^\text{target})$. %Thus, the TD error becomes $\text{TD}(s, a, r, s', d) = r + d\gamma\hat{q}\left(s', \arg\max_{a'}\hat{q}(s', a'; w); w^\text{target}\right) - \hat{q}(s, a; w)$.
Multi-step returns generalize the one-step TD target to bootstrapping over $n$ steps using the $n$-step return $r_t^{(n)} = \sum_{k=0}^{n-1} \gamma^kr_{t+k+1}$. Combined with Double Q-Learning, the TD error becomes $\text{TD}(s_t, a_t, r_{t+1}, s_{t+1}, d) = r_t^{(n)} + d\gamma^n\hat{q}\left(s_{t+n}, \arg\max_{a'}\hat{q}(s_{t+n}, a'; w); w^\text{target}\right) - \hat{q}(s_t, a_t; w)$.

\subsubsection{Noisy Networks}\label{subsubsec:noisy}

Noisy Networks~\cite{fortunato2019noisy} use noisy linear layers instead of regular linear layers, which are of the form $f(x) = (\mu^w + \sigma^w \odot \epsilon^w)x + (\mu^b + \sigma^b \odot \epsilon^b)$ for input $x \in \mathbb{R}^{\text{in}}$ where $\mu^w, \sigma^w \in \mathbb{R}^{\text{out} \times \text{in}}$ and $\mu^b, \sigma^b$ are learnable parameters and $\epsilon^w \in \mathbb{R}^{\text{out} \times \text{in}}, \epsilon^b \in \mathbb{R}^{\text{out}}$ are random variables with $\epsilon^\text{out} \sim \mathcal{N}(0, I) \in \mathbb{R}^{\text{out}}$, $\epsilon^\text{in} \sim \mathcal{N}(0, I) \in \mathbb{R}^{\text{in}}$, $(\epsilon^W)_{ij} = f(\epsilon^\text{out}_i)f(\epsilon^\text{in}_j)$, $(\epsilon^b)_j = f(\epsilon^\text{out}_j)$ where $f(x) = \text{sgn}(x)\sqrt{|x|}$.
Noisy Networks empirically lead to better exploration than $\epsilon$-greedy strategies when the agent has to perform many actions to collect the first non-zero reward in the MDP~\cite{Hessel2018:Rainbow}. They give efficient state-dependent exploration without using $\epsilon$-greedy action selection.

\subsubsection{Distributional Reinforcement Learning}\label{subsubsec:distributional}

Distributional Reinforcement Learning~\cite{bellemare17distributional} aims to tackle the problem of stochastic returns. In particular, for state $s_t$ and action $a_t$, it predicts a discrete distribution of values $\hat{p}(s_t, a_t; w)$ over atoms $z$ on a uniform support between $\text{return}_\text{min}$ and $\text{return}_\text{max}$ instead of a single value $\hat{q}(s_t, a_t; w)$. To train an agent with Distributional Reinforcement Learning, we use a variant of the Bellman optimality equation where, for the optimal policy $\pi^*$, $\hat{p}(s_t, a_t; w)$ (defined on atoms $z$) has to match $\hat{p}(s_{t+1}, \arg\max_{a'} \hat{q}(s_{t+1}, a'; w^\text{target}); w)$ (defined on atoms $r_{t+1} + \gamma z$) when the latter is projected onto atoms $z$ using linear interpolation. Here, $\hat{q}(s_t, a_t; w) = z^\top \hat{p}(s_t, a_t; w)$. The closeness of the two distributions is measured by the Kullback-Leibler divergence. As our values are highly stochastic in the \texttt{laser-hockey} environment when not considering a fixed opponent, we were excited to see how Distributional Reinforcement Learning performs in our setting.

\subsection{Experimental Evaluation}\label{subsec:rainbow_experimental}

\subsubsection{Base Experiments}\label{subsubsec:base}

\begin{figure}
    \centering
    \includegraphics[width=\linewidth]{gfx/basic_experiments_rainbow.pdf}
    \caption{Fraction of games won for different MDP setups and algorithm variants against the weak and strong opponents, respectively. Lines correspond to the mean performance; shaded areas correspond to minimum and maximum performance per time step across three seeds. The experiments are described in detail in~\ref{subsec:rainbow_experimental}.}
    \label{fig:basic_experiments_rainbow}
\end{figure}


In our base experiments, we aimed to understand how the modifications to the MDP detailed in \ref{subsubsec:mdp} affect the learning process and what the effects of certain algorithmic changes are. \textbf{Experiment 1} uses DQN with the original reward $r = \texttt{sparse\_reward}$ $+\ \texttt{reward\_closeness\_to\_puck}$ and prioritized experience replay. \textbf{Experiment 2} adds the advanced reward from~\ref{subsubsec:base}. \textbf{Experiment 3} adds dueling~\cite{wang2016:DDQN} to the architecture. \textbf{Experiment 4} augments the state space according to \ref{subsubsec:state_space}, and \textbf{Experiment 5} uses a set of 24 discrete actions detailed in Appendix~\ref{appendix:rainbow}. Finally, \textbf{Experiment 6} uses observation stacking with four frames (\ref{subsubsec:mdp}). All experiments use the same hyperparameter setup deemed best for the tournament (Appendix~\ref{appendix:rainbow}) and are run on three different seeds.

The results are shown in Fig.~\ref{fig:basic_experiments_rainbow}. \textbf{Experiment 2} highlights that considering a well-designed reward that highly facilitates learning does still not provide a strong enough learning signal for the baseline DQN algorithm. Interestingly, switching to a dueling architecture already allows the agent to learn sensible strategies, highlighting the prominance of dueling for the \texttt{laser-hockey} environment. Training this variant for more steps might also result in perfect convergence. Augmenting the state space leads to a notable improvement in convergence speed, and adding a more granular action space further contributes to the algorithm's stability. Finally, observation stacking also improves convergence and allows the agent to learn much more subtle strategies for the tournament during self-play training.

\subsubsection{Advanced Experiments}\label{subsubsec:advanced}

\begin{figure}
    \centering
    \includegraphics[width=\linewidth]{gfx/advanced_experiments_rainbow.pdf}
    \caption{Average fraction of games won for different Rainbow additions against unseen opponents from \textbf{Experiment 5-6} (left) and against weak and strong basic opponents (right). Lines correspond to the mean performance; shaded areas correspond to minimum and maximum performance per time step across three seeds. The experiments are described in detail in~\ref{subsec:rainbow_experimental}.}
    \label{fig:advanced_experiments_rainbow}
\end{figure}

In our advanced experiments, we started from the dueling DQN baseline with prioritized experience replay from \textbf{Experiment 6} and evaluated the effects of the additional Rainbow components discussed in~\ref{subsec:rainbow_methods}. As half of the base experiments already converged to nearly perfect scores against the weak and strong opponents in six million frames, we set up a more challenging benchmark of beating the agents of \textbf{Experiments 5 and 6} (meaning six agents in total because of using three seeds per experiment). To this end, we trained each agent against the weak and strong opponents until they reached a win rate of 90\% against the strong opponent, then started self-playing by adding a copy of the agent's current state to the set of opponents we train against every 100,000 frames. All trained agents use the hyperparameters in Appendix~\ref{appendix:rainbow} and dueling: the modifications discussed below are \emph{additions} to this baseline. However, unlike \textbf{Experiments 1-6}, the following experiments do not consider cumulative additions to ensure we can understand the effect of additional components independently. For example, \textbf{Experiment 8} does not contain the additions incorporated in \textbf{Experiment 7}.

Fig.~\ref{fig:advanced_experiments_rainbow} shows the average win percentage against the six unseen agents over the course of training and also against the weak and strong opponents for the sake of comparability to Fig.~\ref{fig:basic_experiments_rainbow}. \textbf{Experiment 7} considers the baseline from \textbf{Experiment 6} but with self-playing as described above. \textbf{Experiment 8} adds double Q-learning instead with 3-step returns (\ref{subsubsec:double}). \textbf{Experiment 9} incorporates noisy networks (\ref{subsubsec:noisy}), and \textbf{Experiment 10} uses Distributional Reinforcement Learning (\ref{subsubsec:distributional}). Finally, \textbf{Experiment 11} uses the full Rainbow algorithm. The results show that none of the further additions increase performance considerably using the same hyperparameter base and recommended values for the additional hyperparameters. Distributional dueling DQN with PER is on par with the baseline regarding both convergence speed and performance. This architecture can represent the stochasticity in the \texttt{laser-hockey} environment well, therefore it is clear why this addition resulted in a very slight improvement in convergence speed. However, it takes considerably more time to run this architecture for 12 million time steps, thus we decided against using it in the tournament where every second counts for fine-tuning the agent on the collected data. The third-best variant is noisy dueling DQN with PER (\textbf{Experiment 9}). As discussed in \ref{subsubsec:noisy}, noisy networks are particularly useful when the agent must perform many correct actions to collect the first non-zero reward, which is the case for, e.g., exploration-based Atari games, but the \texttt{laser-hockey} environment does not require such complex long-term planning. Double Q-learning made the agent more unstable and did not lead to faster convergence. A notable problem with multi-step returns is that they make the algorithm on-policy: the \emph{future} rewards collected according to the noisy policy are also incorporated into the TD target. We did not find this addition useful for the \texttt{laser-hockey} environment. Finally, the full Rainbow algorithm failed to converge using the suggested hyperparameters in~\cite{Hessel2018:Rainbow} for the individual components. It was also by far the slowest variant of DQN.

\subsubsection{Limitations}

The number of experiments conducted made it infeasible to find a hyperparameter setup for each experiment that works best. We aimed to find a unified hyperparameter base that is stable across experiments and is performant for the final version we used for the tournament: the dueling DQN with prioritized experience replay. However, one might find even better hyperparameter setups for the individual experiments in a much more large-scale experiment that could influence the obtained results. Apart from the learning rate, we stayed quite close to the hyperparameters suggested in~\cite{Hessel2018:Rainbow} for all experiments (Appendix~\ref{appendix:rainbow}).


\section{Soft Actor-Critic (SAC)}
 SAC algorithm is an entropy-based off-policy learning algorithm for continuous control problem. To understand SAC algorithm better, we must first dive into the \textit{maximum-entropy RL} framework.

\textbf{Maximum Entropy Reinforcement Learning:} In conventional RL, the agent's motivation is to maximize the cumulative sum of the expected reward \cite{Sutton1998}. Maximum Entropy RL augments this objective function incorporating a term encouraging exploration \cite{max_ent}. According to this framework an optimal policy is expected to maximize reward while being as stochastic as possible. $ \pi^*=\arg \max _\pi \mathcal{J}(\pi) =\arg \max _\pi \sum_t \mathbb{E}_{\left(\mathbf{s}_t, \mathbf{a}_t\right) \sim \rho_\pi}\left[r\left(\mathbf{s}_t, \mathbf{a}_t\right)+\alpha \mathcal{H}\left(\pi\left(\cdot \mid \mathbf{s}_t\right)\right)\right],$ where  $\rho_\pi$ represents the probability distribution the policy, and $\mathcal{H}$ represents the entropy of the stochastic policy $\pi$ given state $s_{t}$. The temperature parameter $\alpha$ scales the randomness of the agent. For the case of $\alpha=0$ the objective originated from the reward hypothesis is obtained. 

Maximum Entropy formulation of the objective function brings about some advantages. Encouraged exploration is helpful to cure the problem of sticking to sub-optimal policies. This is also good for agent's containment of equally eligible multiple different policies. 

\textbf{Soft Policy Iteration: } Soft Policy iteration indicates the process of alternating between policy evaluation and policy improvement during learning maximum entropy policies. It is the corresponding concept name of policy iteration in maximum entropy RL framework \cite{HaarnojaAbbeelLevine2018:SAC}.  

In policy evaluation phase, our aim is to compute the value of a soft policy according to the maximum entropy objective. Therefore, we need to compute the soft Q-value, $Q\!:\!\mathcal{S}\text{x}\mathcal{A}\!\rightarrow\!\mathbb{R}$, for a fixed policy. This can be done via starting with any Q function and iteratively applying a modified version of Bellman update operation $\mathcal{T}^\pi$: $ \mathcal{T}^\pi Q\left(\mathbf{s}_t, \mathbf{a}_t\right) \triangleq r\left(\mathbf{s}_t, \mathbf{a}_t\right)+\gamma \mathbb{E}_{\mathbf{s}_{t+1} \sim p}\left[V\left(\mathbf{s}_{t+1}\right)\right] $ where $    V\left(\mathbf{s}_t\right)=\mathbb{E}_{\mathbf{a}_t \sim \pi}\left[Q\left(\mathbf{s}_t, \mathbf{a}_t\right)-\log \pi\left(\mathbf{a}_t \mid \mathbf{s}_t\right)\right].
\label{equation:soft_policy_bellman}$ In former equation $V\!:\!\mathcal{S}\!\rightarrow\!\mathbb{R}$ indicates the soft value function. The remaining step of the soft policy iteration is policy improvement. For policy improvement, the set of policies $\Pi$ is restricted to parameterized family of distributions. This is a way of ensuring tractability, necessary for gradient flow in the learning process. The policy is improved via an update towards the exponential of the new Q-function. The authors used Kullback-Leibler divergence for updating current policy: $\pi_{\text {new }}=\arg \min _{\pi^{\prime} \in \Pi} \mathrm{D}_{\mathrm{KL}}\left(\pi^{\prime}\left(\cdot \mid \mathbf{s}_t\right) \bigg| \bigg| \frac{\exp \left(Q^{\pi_{\text {old }}}\left(\mathbf{s}_t, \cdot\right)\right)}{Z^{\pi_{\text {old }}}\left(\mathbf{s}_t\right)}\right).
$\label{eqn:soft_policy_kl} The partition function $Z^{\pi_{\mathrm{old}}}\left(\mathbf{s}_t\right)$ is for normalizing the probability distribution. It does not contribute to the gradient of the divergence with respect to policy parameters. 

This soft policy iteration algorithm, policy improvement and evaluation, is guaranteed to converge to the optimal maximum entropy policy under finite action space assumption and restricted policy domain $\Pi$. The convergence proofs are skipped for the sake of space, can be seen in \cite{HaarnojaAbbeelLevine2018:SAC}. 

\textbf{SAC Algorithm:} Previous sections were crucial to understand Soft Actor-Critic algorithm. For extending the learning process to continuous domains, authors leveraged neural networks as policy and state value function approximators \cite{HaarnojaAbbeelLevine2018:SAC}. In first version of the paper \cite{HaarnojaAbbeelLevine2018:SAC} they conceived a parameterized soft Q function $Q_{\theta}(s_t,a_t)$, value function $V_\psi(s_t)$, and a tractable policy $\pi_{\phi}(a_t \mid s_t)$, and target networks for value and critic networks $\bar{\psi}, \bar{\theta}$. The soft value function is updated through mean square error loss on sampled transactions from replay buffer $\mathcal{D}$: $J_V(\psi)=\mathbb{E}_{\mathbf{s}_t \sim \mathcal{D}}\left[\frac{1}{2}\left(V_\psi\left(\mathbf{s}_t\right)-\mathbb{E}_{\mathbf{a}_t \sim \pi_\phi}\left[Q_\theta\left(\mathbf{s}_t, \mathbf{a}_t\right)-\log \pi_\phi\left(\mathbf{a}_t \mid \mathbf{s}_t\right)\right]\right)^2\right].$ 
Parallel to that, the critic is updated through the state value approximated by target value network: $J_Q(\theta)=\mathbb{E}_{\left(\mathbf{s}_t, \mathbf{a}_t\right) \sim \mathcal{D}}\left[\frac{1}{2}\left(Q_\theta\left(\mathbf{s}_t, \mathbf{a}_t\right)- 
    r\left(\mathbf{s}_t, \mathbf{a}_t\right)-\gamma \mathbb{E}_{\mathbf{s}_{t+1} \sim p}\left[V_{\bar{\psi}}\left(\mathbf{s}_{t+1}\right)\right]
    \right)^2\right].$ \label{critic_update:sac} The policy network is used to regress a Gaussian distribution and sample actions from it: $a_{t}=f_{\phi}(\epsilon_t; s_t)$ where $\epsilon_{t}$ is a noise sampled from univariate Normal distribution. The policy network is trained via reparameterization trick proposed in \cite{kingma2022autoencoding}. The update is done to minimize the expected KL divergence in equation \ref{eqn:soft_policy_kl}. Therefore, the gradient can be approximated with: $\nabla_\phi \hat{J}_\pi(\phi)=\nabla_\phi \alpha \log \pi_\phi\left(\mathbf{a}_t \mid \mathbf{s}_t\right)+\left(\nabla_{\mathbf{a}_t} \alpha \log \pi_\phi\left(\mathbf{a}_t \mid \mathbf{s}_t\right)-\nabla_{\mathbf{a}_t} Q\left(\mathbf{s}_t, \mathbf{a}_t\right)\right) \nabla_\phi\left(\boldsymbol{\epsilon}_t ; \mathbf{s}_t\right)$. The target networks are updated through soft updates. The next parts are for different variants of SAC algorithm.

\textbf{Auto-Tuning Temperature Hyperparameter $\alpha$:} The temperature hyper-parameter $\alpha$ is introduced to balance between exploration and exploitation \cite{HaarnojaAbbeelLevine2018:SAC}. This introduction was adding another hyper-parameter to tune across different environments. Additionally, this was also harming agent's scaling to different reward functions across different environments. In their follow-up work, they introduce a mechanism to update the temperature parameter $\alpha$ \cite{haarnoja2019soft}. This formula is obtained through solving the dual problem of maximizing entropy objective, 
Update formula and its proof is skipped due to space constraints. We implemented this update mechanism and observed its effects in section \ref{sac_results}.

\textbf{Modifying Value Network:} In the first version of the paper \cite{HaarnojaAbbeelLevine2018:SAC} authors employed a value and a critic network to be used in updating the actor. However it has been demonstrated that single critic network is susceptible to overestimation of the Q values \cite{lillicrap16ddpg}. This issue could be relieved with having two critic networks, minimum of which is taken for actor update \cite{fujimoto2018:TD3}. Two critic version takes place in a follow-up work \cite{haarnoja2019soft}. We also implemented the version with two critic and observed the results. 

\textbf{SR-SAC:} Recently it has been shown that the sample efficiency of the RL algorithms could be improved with increasing replay ratio under carefully designed conditions \cite{d'oro2023sampleefficient}. They have shown the effectiveness of resetting actor and critic networks in relatively high replay ratio settings. The motivation behind resetting is to eliminate the primacy bias, that is the agent's tendency to be affected by the formerly collected data \cite{nikishin2022primacy}. 

\subsection{Experimental evaluation \& Discussion}
\label{sac_results}

This section is for the experiments in above-mentioned settings. Hyper-parameters for SAC agent is tuned through manual search. For the best agent we have a learning rate of $3*10^{-4}$, buffer size of 300K, two hidden dimensions containing 256 hidden neurons. All settings except \textbf{Setting6}(S6) uses two critics for value estimation.

Upper part of Figure \ref{fig:perf_plot_sac} displays the performances of different SAC agents against weak and strong agents. \textbf{S0} is for default state and augmented reward, 
\textbf{S1} is for augmented state and  reward, \textbf{S2} is for default state and reward, \textbf{S3} is for augmented state and default reward. Among those setting we see that both augmentations are beneficial for learning. For the remaining settings we continue with augmented state and reward. \textbf{S4} is for Auto-Tuning $\alpha$, \textbf{S5} is for fixed $\alpha=0.1$, \textbf{S6} is for the vanilla version using a value and a critic network, \textbf{S7} is for scaling by reset version. We see that there is an advantage to using two critics instead of vanilla version. This is because of more accurate value estimation. Algorithm best performs with lower $\alpha$ values.  

\begin{figure}
    \centering
    \includegraphics[width=\linewidth]{gfx/perf_plot_sac.pdf}
    \includegraphics[width=\linewidth]{gfx/sr_plot_sac.pdf}
    \caption{Upper: Performance of SAC agent trained in different settings against weak and strong agent, Lower: Effect of scaling by resetting in different settings against weak and strong agent}
    \label{fig:perf_plot_sac}
\end{figure}

Lower part of Figure  \ref{fig:perf_plot_sac} displays the effect of scaling by reset across different replay ratios. For implementation of SR-SAC the inverval of resetting is every 0.5M gradient steps of the networks. \textbf{S0} is for scaling with replay ratio of 1, \textbf{S1} is for scaling with replay ratio of 2, \textbf{S2} is for scaling with replay ratio of 4, \textbf{S3} is default version with replay ratio 1, \textbf{S4} is default version with replay ratio 2, \textbf{S5} is default version with replay ratio 4. We see that reset by scaling deteriorates performance. This may be the problem caused by the not finding an optimal interval value for resetting. On the other hand, for the default version it is beneficial to increase the replay ratio.

 


 


\section{Conclusion}

In our report, we aimed to compare three notably different off-policy reinforcement learning algorithms in the \texttt{laser-hockey} environment. According to our results, not all of the components of the Rainbow algorithm led to a performance increase but there were beneficial ones that allowed the agent to be the second-best player in the tournament. The discussed modifications to the MDP also led to considerable improvements and were the core reason of the algorithm's success. We can observe that discretizing the action space does \emph{not} hurt performance when the action space is chosen carefully. This resolved our main concerns with DQN-based agents.

TD3 has proven to be a strong approach in this dynamic environment with a continuous action space. It was able to consistently learn to defeat the two \texttt{BasicOpponents} in a small number of steps. Surprisingly, pink action noise led to a significant decrease in performance and convergence speed across all experiments. A much harder task has been to not only win against seen agents, but also generalize to unknown opponents. For this, self-training and in particular playing against other previously trained, strong agents has been tremendously helpful.

All of these experiments and approaches have helped us to enter the two by far best model-free reinforcement learning agents to the final tournament (Rainbow and TD3), claiming a strong second place behind \texttt{MuZero}.

\bibliographystyle{abbrv}
\bibliography{main}

\appendix

\newpage\null
%\thispagestyle{empty}\newpage

\section{Appendix TD3}

\subsection{Hyperparameters}

\begin{table}[h]
    \centering
    \caption{Hyperparameters that were used for all runs with TD3 unless mentioned otherwise in the report.}
    \label{tab:td3_params}
    \begin{tabular}{|l|l|}
    \hline
         \textbf{Name} & \textbf{Value} \\ \hline
         Architecture & 256 $\times$ 256 \\
         Activation hidden & ReLU \\
         Activation actor & TanH \\
         Reward & \texttt{Weighted} \\
         Discount factor $\gamma$ & $0.95$ \\
         Batch size $B$ & 128 \\
         Learning starts & 50,000 \\
         Target policy noise $\Tilde{\sigma}$ & 0.2 \\
         Target noise clip $c$ & 0.5 \\
         Action noise $\sigma$ & 0.1 \\
         Policy delay $d$ & $2$ \\
         Polyak parameter $\tau$ & $5 \cdot 10^{-3}$ \\
         Learning rate critics & $10^{-4}$ \\
         Learning rate actor & $10^{-4}$ \\
         Buffer size base & $200,000$ \\
         Training steps base N & $3,000,000$ \\
         Buffer size advanced & $1,000,000$ \\
         Training steps advanced N & $12,000,000$ \\
    \hline
    \end{tabular}
\end{table}

\subsection{Full Ablation Study}

\begin{table}[h]
    \centering
    \caption{Full ablation study of TD3 including three architectures and activation functions. The results show the average win percentages and standard deviations across five seeds against the weak and strong \texttt{BasicOpponents}.}
    \label{tab:td3_base_experiment_full}
    \begin{tabular}{|c|c|c|c||l|l|} 
    \hline
    \textbf{Reward} & \textbf{Architecture} & \textbf{Activation} & \textbf{State Augm.} & \textbf{Weak Opponent} & \textbf{Strong Opponent} \\ \hline
    Sparse & $256 \times 256$ & ReLU & / & $38.90\% \pm 22.19$ & $33.14\% \pm 15.75$ \\ 
    Weighted & $256 \times 256$ & ReLU & / & $89.02\% \pm 20.37$ & $89.19\% \pm 19.05$ \\ 
    Weighted & $128 \times 128$ & ReLU & / & $86.80\% \pm 17.28 $ & $85.68\% \pm 16.19$ \\ 
    Weighted & $400 \times 300$ & ReLU & / & $97.22\% \pm  0.33$ & $85.12\% \pm 27.02$ \\
    Weighted & $256 \times 256$ & Mish & / & $90.32\% \pm 10.61$ & $61.76\% \pm 11.65$ \\
    Weighted & $256 \times 256$ & ReLU & \texttt{distance} & $\mathbf{99.06\% \pm 0.79}$ & $\mathbf{98.22\% \pm 0.97}$ \\
    \hline
    \end{tabular}
\end{table}

The full ablation study for TD3 can be found in Table~\ref{tab:td3_base_experiment_full}. All experiments used neural networks with two hidden, fully-connected layers for the actor and critics. We tested three architectures with $(128 \times 128)$ neurons, $(256 \times 256)$ and $(400 \times 300)$. The latter two perform both well on average, while the lower capacity one is a bit worse. We preferred $(256 \times 256)$, because it has the better win percentage against the strong opponent. 

Next, we compared the previous best against the same approach, but with Mish instead of ReLU as activation \cite{misra2020MishAS}. Mish is a ReLU-like activation function that has been shown to consistently outperform ReLU and similar activation functions in computer vision tasks. In the \texttt{laser-hockey} environment we were not able to replicate these results and continued using ReLU.

\subsection{Upper Confidence Bound}\label{subsec:ucb}

Thanks to our extensive experiments, we ended up with a large number of agents that were candidates to be the final agent in the tournament. Unfortunately, selecting the best agent is very difficult, because the performance against the opponents is unknown before playing against them in the tournament. For this reason, we implemented the Upper Confidence Bound algorithm (UCB) \cite{agrawal1995SampleMB}. 

In UCB, each agent $a_i$ of an ensemble is treated as a bandit that returns a reward drawn from a distribution $P_i$ upon being chosen as action $A_t$ in game $t$. The reward in our setting is the \texttt{winner} information. UCB chooses the bandit as the next action that has the highest expected reward, that is still realistic. In each game $t$ the agent is chosen as $A_t = \operatorname{argmax}_i \hat{\mu}_i(t-1) + \sqrt{\frac{2 \cdot \sigma^2 \log(t^3)}{T_i(t-1)}}$, where $\hat{\mu}_i(t-1)$ is the empirical mean of the rewards of agent $a_i$ in the first $t-1$ games, $\sigma$ controls the trade-off between exploration and exploitation and $T_i$ is the amount of times agent $a_i$ has been played. 

%Over time, the UCB increases relatively for agents with a higher win percentage and agents that have not been played often.


\section{Appendix Rainbow}\label{appendix:rainbow}

\subsection{Discretized Actions}

\begin{table}[H]
\centering
\caption{Basic discrete action space.}
\begin{tabular}{cccc}
\toprule
\textbf{Linear force $x$} & \textbf{Linear force $y$} & \textbf{Torque} & \textbf{Shooting} \\
\midrule
-1 & 0 & 0 & 0 \\
1 & 0 & 0 & 0 \\
0 & -1 & 0 & 0 \\
0 & 1 & 0 & 0 \\
0 & 0 & -1 & 0 \\
0 & 0 & 1 & 0 \\
0 & 0 & 0 & 1 \\
0 & 0 & 0 & 0 \\
\bottomrule
\end{tabular}
\end{table}

\begin{table}[H]
\centering
\caption{Advanced discrete action space.}
\begin{tabular}{cccc}
\toprule
\textbf{Linear force $x$} & \textbf{Linear force $y$} & \textbf{Torque} & \textbf{Shooting} \\
\midrule
0 & 0 & 0 & 0 \\
-1 & 0 & 0 & 0 \\
1 & 0 & 0 & 0 \\
0 & -1 & 0 & 0 \\
0 & 1 & 0 & 0 \\
0 & 0 & -1 & 0 \\
0 & 0 & 1 & 0 \\
-1 & -1 & 0 & 0 \\
-1 & 1 & 0 & 0 \\
1 & -1 & 0 & 0 \\
1 & 1 & 0 & 0 \\
-1 & -1 & -1 & 0 \\
-1 & -1 & 1 & 0 \\
-1 & 1 & -1 & 0 \\
-1 & 1 & 1 & 0 \\
1 & -1 & -1 & 0 \\
1 & -1 & 1 & 0 \\
1 & 1 & -1 & 0 \\
1 & 1 & 1 & 0 \\
0 & 0 & 0 & 1 \\
0 & -1 & -1 & 0 \\
0 & -1 & 1 & 0 \\
0 & 1 & -1 & 0 \\
0 & 1 & 1 & 0 \\
\bottomrule
\end{tabular}
\end{table}

\subsection{Hyperparameters}

\begin{table}[H]
    \centering
    \caption{Agent hyperparameters used across the experiments. Set notation shows the set of values used in the experiments. $v_\text{min}$, $v_\text{max}$, and atom\_size are hyperparameters of the distributional approach.}
    \begin{tabular}{ll}
    \toprule
    \textbf{Parameter} & \textbf{Value} \\
    \midrule
    $v_\text{min}$ & -15 \\
    $v_\text{max}$ & 15 \\
    atom size & 51 \\
    states dimensionality & \{18, 27\} \\
    hidden layers & [512] \\
    number of actions & \{8, 24\} \\
    number of stacked observations & \{1, 4\} \\
    activation function & ReLU \\
    reward & \{baseline, ours\} \\
    observation & \{baseline, augmented\} \\
    \bottomrule
    \end{tabular}
\end{table}

\begin{table}[H]
    \centering
    \caption{Algorithm hyperparameters. Set notation shows the set of values used in the experiments.}
    \begin{tabular}{ll}
    \toprule
    \textbf{Parameter} & \textbf{Value} \\
    \midrule
    batch size & 32 \\
    max gradient norm & 10 \\
    multi-step value & \{1, 3\} \\
    train frequency & 4 \\
    learning starts & 80,000 \\
    Polyak $\tau$ & 1 \\
    target update frequency & 1000 \\
    max number of timesteps/episode & 250 \\
    $\epsilon_\text{initial}$ & 1 \\
    $\epsilon_\text{final}$ & 0.1 \\
    exploration fraction & 0.0375 \\
    $\gamma$ & 0.95 \\
    total timesteps & 12,000,000 \\
    number of evaluation episodes & 1000 \\
    start pretrained delta & 500,000 \\
    start self-play threshold & 0.9 \\
    \bottomrule
    \end{tabular}
\end{table}

\begin{table}[H]
    \centering
    \caption{Buffer hyperparameters.}
    \begin{tabular}{ll}
    \toprule
    \textbf{Parameter} & \textbf{Value} \\
    \midrule
    $\alpha$ & 0.5 \\
    $\beta_\text{start}$ & 0.4 \\
    $\beta_\text{end}$ & 1 \\
    buffer size & 500,000 \\
    \bottomrule
    \end{tabular}
\end{table}

\begin{table}[H]
    \centering
    \caption{Optimizer hyperparameters.}
    \begin{tabular}{ll}
    \toprule
    \textbf{Parameter} & \textbf{Value} \\
    \midrule
    optimizer & Adam~\cite{kingma2017adam} \\
    learning rate & 0.0001 \\
    scheduler steps & [625,000; 1,000,000] \\
    scheduler multiplicative factor & 0.5 \\
    \bottomrule
    \end{tabular}
\end{table}

\subsection{Tournament Architecture and Training}

Considering the results in~\ref{subsubsec:advanced}, we decided to use a \textbf{dueling DQN architecture with prioritized experience replay} for the tournament. The algorithm was trained with self-training and also against previous strong agents of the team for 6-12 million steps, using multiple seeds and periodic evaluation against unseen agents. This procedure was repeated 4-5 times, starting from promising previous checkpoints and always including the previous best agents of the team (TD3 and SAC). The final model was also trained on the tournament server by using the previously described procedure but loading a random fraction of the tournament data every 100,000 steps to mix it with the transitions collected by the off-policy agent. Mixing the observations was crucial: fine-tuning only on tournament data deteriorated performance. Our final agent became the second-best agent in the entire tournament according to the leaderboard.

\end{document}